%%%%%%%%%%%%%%%%%%%%%%%%%%%%%%%%%%%%%%%%%
% Structured General Purpose Assignment
% LaTeX Template
%
% This template has been downloaded from:
% http://www.latextemplates.com
%
% Original author:
% Ted Pavlic (http://www.tedpavlic.com)
%
% Note:
% The \lipsum[#] commands throughout this template generate dummy text
% to fill the template out. These commands should all be removed when 
% writing assignment content.
%
%%%%%%%%%%%%%%%%%%%%%%%%%%%%%%%%%%%%%%%%%

%----------------------------------------------------------------------------------------
%	PACKAGES AND OTHER DOCUMENT CONFIGURATIONS
%----------------------------------------------------------------------------------------

\documentclass{article}

\usepackage{fancyhdr} % Required for custom headers
\usepackage{lastpage} % Required to determine the last page for the footer
\usepackage{extramarks} % Required for headers and footers
\usepackage{graphicx} % Required to insert images
\usepackage{lipsum} % Used for inserting dummy 'Lorem ipsum' text into the template
\usepackage{enumerate}
\usepackage{booktabs}
\usepackage{amsmath}

% Margins
\topmargin=-0.45in
\evensidemargin=0in
\oddsidemargin=0in
\textwidth=6.5in
\textheight=9.0in
\headsep=0.25in 

\linespread{1.5} % Line spacing

% Set up the header and footer
\pagestyle{fancy}
\lhead{\hmwkAuthorName} % Top left header
\chead{\hmwkClass\ (\hmwkTitle)} % Top center header
%%\rhead{\firstxmark} 
\rhead{} % Top right header
\lfoot{\lastxmark} % Bottom left footer
\cfoot{} % Bottom center footer
\rfoot{Page\ \thepage\ of\ \pageref{LastPage}} % Bottom right footer
\renewcommand\headrulewidth{0.4pt} % Size of the header rule
\renewcommand\footrulewidth{0.4pt} % Size of the footer rule

\setlength\parindent{0pt} % Removes all indentation from paragraphs

%----------------------------------------------------------------------------------------
%	DOCUMENT STRUCTURE COMMANDS
%	Skip this unless you know what you're doing
%----------------------------------------------------------------------------------------

% Header and footer for when a page split occurs within a problem environment
\newcommand{\enterProblemHeader}[1]{
\nobreak\extramarks{#1}{#1 continued on next page\ldots}\nobreak
\nobreak\extramarks{#1 (continued)}{#1 continued on next page\ldots}\nobreak
}

% Header and footer for when a page split occurs between problem environments
\newcommand{\exitProblemHeader}[1]{
\nobreak\extramarks{#1 (continued)}{#1 continued on next page\ldots}\nobreak
\nobreak\extramarks{#1}{}\nobreak
}

\setcounter{secnumdepth}{0} % Removes default section numbers
\newcounter{homeworkProblemCounter} % Creates a counter to keep track of the number of problems

\newcommand{\homeworkProblemName}{}
\newenvironment{homeworkProblem}[1][Problem \arabic{homeworkProblemCounter}]{ % Makes a new environment called homeworkProblem which takes 1 argument (custom name) but the default is "Problem #"
\stepcounter{homeworkProblemCounter} % Increase counter for number of problems
\renewcommand{\homeworkProblemName}{#1} % Assign \homeworkProblemName the name of the problem
\section{\homeworkProblemName} % Make a section in the document with the custom problem count
\enterProblemHeader{\homeworkProblemName} % Header and footer within the environment
}{
\exitProblemHeader{\homeworkProblemName} % Header and footer after the environment
}

\newcommand{\problemAnswer}[1]{ % Defines the problem answer command with the content as the only argument
\noindent\framebox[\columnwidth][c]{\begin{minipage}{0.98\columnwidth}#1\end{minipage}} % Makes the box around the problem answer and puts the content inside
}

\newcommand{\homeworkSectionName}{}
\newenvironment{homeworkSection}[1]{ % New environment for sections within homework problems, takes 1 argument - the name of the section
\renewcommand{\homeworkSectionName}{#1} % Assign \homeworkSectionName to the name of the section from the environment argument
\subsection{\homeworkSectionName} % Make a subsection with the custom name of the subsection
\enterProblemHeader{\homeworkProblemName\ [\homeworkSectionName]} % Header and footer within the environment
}{
\enterProblemHeader{\homeworkProblemName} % Header and footer after the environment
}
   
%----------------------------------------------------------------------------------------
%	NAME AND CLASS SECTION
%----------------------------------------------------------------------------------------

\newcommand{\hmwkTitle}{Homework\ \#3} % Assignment title
\newcommand{\hmwkDueDate}{Tuessday,\ February\ 13,\ 2018} % Due date
\newcommand{\hmwkClass}{FIN\ 513} % Course/class
\newcommand{\hmwkClassTime}{9:30am} % Class/lecture time
\newcommand{\hmwkAuthorName}{Wanbae Park} % Your name

%----------------------------------------------------------------------------------------
%	TITLE PAGE
%----------------------------------------------------------------------------------------

\title{
\vspace{2in}
\textmd{\textbf{\hmwkClass:\ \hmwkTitle}}\\
\normalsize\vspace{0.1in}\small{Due\ on\ \hmwkDueDate}\\
\vspace{3in}
}

\author{\textbf{\hmwkAuthorName}}
\date{} % Insert date here if you want it to appear below your name

%----------------------------------------------------------------------------------------

\begin{document}

\maketitle

%----------------------------------------------------------------------------------------
%	TABLE OF CONTENTS
%----------------------------------------------------------------------------------------

%\setcounter{tocdepth}{1} % Uncomment this line if you don't want subsections listed in the ToC

%%\newpage
%%\tableofcontents
\newpage

%----------------------------------------------------------------------------------------
%	PROBLEM 1
%----------------------------------------------------------------------------------------

% To have just one problem per page, simply put a \clearpage after each problem

\begin{homeworkProblem}
	The put-call parity of American options on a currency is $e^{-r_f(T - t)}S - K \leq C - P \leq S - e^{-r_d(T-t)}K$ or $B^f S - K \leq C - P \leq S - B^dK$, where $S$ and $K$ denotes spot exchange rate and strike price as usual, $r_d$ and $r_f$ denotes domestic risk-free rate and foreign risk-free rate, respectively.	\\
	\\
	\textit{(Proof)} Suppose not.
	\begin{enumerate}
		\item Suppose $C - P > S - e^{-r_d(T - t)}K$. Then an arbitrage opportunity exists by constructing the following portfolio.
		\begin{enumerate}
			\item Buy a put option and a unit of foreign currency.
			\item Write a call option and borrow $e^{-r_d(T - t)}K$ amount of domestic currency on risk-free rate.
		\end{enumerate}
		All possible situations can be separated by two cases: early exercise of written option occurs and no early exercise of written option. Assuming that early exercise occurs at $t^* < T$, payoff of the portfolio is as follows.
		\begin{enumerate}[\text{\textit{Case}} 1.]
		\item Early exercise at $t^* < T$.
			\begin{enumerate}[(a)]
				\item $e^{r_f(t^* - t)}S_{t^*}$
				\item $-(S_{t^*} - K)  - e^{-r_d(T - t^*)}K$
			\end{enumerate}
			The sum of payoff from two strategies is $(e^{r_f(t^* - t)} - 1)S_{t^*} + (1 - e^{-r_d(T - t^*)})K$, which is positive.
		\item No early exercise
			\begin{enumerate}
				\item $S_T > K$
				\begin{enumerate}[(a)]
					\item $0 + e^{r_f(T - t)}S_T$
					\item $-(S_T - K) - K$
				\end{enumerate}
				\item $S_T \leq K$
				\begin{enumerate}[(a)]
					\item $(K - S_T) + e^{r_f(T - t)}S_T$
					\item $0 - K$
				\end{enumerate}
			\end{enumerate}
			In this case, the portfolio also has a positive payoff regardless of spot exchange rate at maturity.
		\end{enumerate}
		Since the portfolio has a positive payoff at all possible situations, there must be a cost for implementing the strategies if there is no arbitrage opportunity. However, by the assumption, the initial cost for constructing portfolio is negative, so a contradiction occurs. Therefore, $C - P \leq S - e^{-r_d(T - t)}K$ must hold.
%----------------------------------------------------------------------------------------
		\item Suppose $C - P < e^{-r_f(T - t)}S - K$. Then there is also an arbitrage opportunity exists considering the following portfolio.
		\begin{enumerate}
			\item Buy a call option and invest K amount of domestic currencies on domestic risk-free rate.
			\item Write a put option and sell short a foreign risk-free zero coupon bond.
		\end{enumerate}
	\end{enumerate}
\end{homeworkProblem}
%----------------------------------------------------------------------------------------
%	PROBLEM 2
%----------------------------------------------------------------------------------------
\begin{homeworkProblem}
\begin{enumerate}[(a)]
	\item
	\item
	\item
	\item
	\item
	\item
	\item
	\item
\end{enumerate}
\end{homeworkProblem}
%----------------------------------------------------------------------------------------
%	PROBLEM 3
%----------------------------------------------------------------------------------------
\begin{homeworkProblem}
\begin{enumerate}[(a)]
	\item
	\item
	\item
	\item
\end{enumerate}
\end{homeworkProblem}
%----------------------------------------------------------------------------------------
%	PROBLEM 4
%----------------------------------------------------------------------------------------
\begin{homeworkProblem}
\begin{enumerate}[(a)]
	\item
	\item
\end{enumerate}
\end{homeworkProblem}
\end{document}
