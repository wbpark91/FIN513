%%%%%%%%%%%%%%%%%%%%%%%%%%%%%%%%%%%%%%%%%
% Structured General Purpose Assignment
% LaTeX Template
%
% This template has been downloaded from:
% http://www.latextemplates.com
%
% Original author:
% Ted Pavlic (http://www.tedpavlic.com)
%
% Note:
% The \lipsum[#] commands throughout this template generate dummy text
% to fill the template out. These commands should all be removed when 
% writing assignment content.
%
%%%%%%%%%%%%%%%%%%%%%%%%%%%%%%%%%%%%%%%%%

%----------------------------------------------------------------------------------------
%	PACKAGES AND OTHER DOCUMENT CONFIGURATIONS
%----------------------------------------------------------------------------------------

\documentclass{article}

\usepackage{fancyhdr} % Required for custom headers
\usepackage{lastpage} % Required to determine the last page for the footer
\usepackage{extramarks} % Required for headers and footers
\usepackage{graphicx} % Required to insert images
\usepackage{lipsum} % Used for inserting dummy 'Lorem ipsum' text into the template
\usepackage{enumerate}
\usepackage{booktabs}
\usepackage{amsmath}

% Margins
\topmargin=-0.45in
\evensidemargin=0in
\oddsidemargin=0in
\textwidth=6.5in
\textheight=9.0in
\headsep=0.25in 

\linespread{1.5} % Line spacing

% Set up the header and footer
\pagestyle{fancy}
\lhead{\hmwkAuthorName} % Top left header
\chead{\hmwkClass\ (\hmwkTitle)} % Top center header
\rhead{\firstxmark} % Top right header
\lfoot{\lastxmark} % Bottom left footer
\cfoot{} % Bottom center footer
\rfoot{Page\ \thepage\ of\ \pageref{LastPage}} % Bottom right footer
\renewcommand\headrulewidth{0.4pt} % Size of the header rule
\renewcommand\footrulewidth{0.4pt} % Size of the footer rule

\setlength\parindent{0pt} % Removes all indentation from paragraphs

%----------------------------------------------------------------------------------------
%	DOCUMENT STRUCTURE COMMANDS
%	Skip this unless you know what you're doing
%----------------------------------------------------------------------------------------

% Header and footer for when a page split occurs within a problem environment
\newcommand{\enterProblemHeader}[1]{
\nobreak\extramarks{#1}{#1 continued on next page\ldots}\nobreak
\nobreak\extramarks{#1 (continued)}{#1 continued on next page\ldots}\nobreak
}

% Header and footer for when a page split occurs between problem environments
\newcommand{\exitProblemHeader}[1]{
\nobreak\extramarks{#1 (continued)}{#1 continued on next page\ldots}\nobreak
\nobreak\extramarks{#1}{}\nobreak
}

\setcounter{secnumdepth}{0} % Removes default section numbers
\newcounter{homeworkProblemCounter} % Creates a counter to keep track of the number of problems

\newcommand{\homeworkProblemName}{}
\newenvironment{homeworkProblem}[1][Problem \arabic{homeworkProblemCounter}]{ % Makes a new environment called homeworkProblem which takes 1 argument (custom name) but the default is "Problem #"
\stepcounter{homeworkProblemCounter} % Increase counter for number of problems
\renewcommand{\homeworkProblemName}{#1} % Assign \homeworkProblemName the name of the problem
\section{\homeworkProblemName} % Make a section in the document with the custom problem count
\enterProblemHeader{\homeworkProblemName} % Header and footer within the environment
}{
\exitProblemHeader{\homeworkProblemName} % Header and footer after the environment
}

\newcommand{\problemAnswer}[1]{ % Defines the problem answer command with the content as the only argument
\noindent\framebox[\columnwidth][c]{\begin{minipage}{0.98\columnwidth}#1\end{minipage}} % Makes the box around the problem answer and puts the content inside
}

\newcommand{\homeworkSectionName}{}
\newenvironment{homeworkSection}[1]{ % New environment for sections within homework problems, takes 1 argument - the name of the section
\renewcommand{\homeworkSectionName}{#1} % Assign \homeworkSectionName to the name of the section from the environment argument
\subsection{\homeworkSectionName} % Make a subsection with the custom name of the subsection
\enterProblemHeader{\homeworkProblemName\ [\homeworkSectionName]} % Header and footer within the environment
}{
\enterProblemHeader{\homeworkProblemName} % Header and footer after the environment
}
   
%----------------------------------------------------------------------------------------
%	NAME AND CLASS SECTION
%----------------------------------------------------------------------------------------

\newcommand{\hmwkTitle}{Homework\ \#2} % Assignment title
\newcommand{\hmwkDueDate}{Thursday,\ February\ 1,\ 2018} % Due date
\newcommand{\hmwkClass}{FIN\ 513} % Course/class
\newcommand{\hmwkClassTime}{9:30am} % Class/lecture time
\newcommand{\hmwkAuthorName}{Wanbae Park} % Your name

%----------------------------------------------------------------------------------------
%	TITLE PAGE
%----------------------------------------------------------------------------------------

\title{
\vspace{2in}
\textmd{\textbf{\hmwkClass:\ \hmwkTitle}}\\
\normalsize\vspace{0.1in}\small{Due\ on\ \hmwkDueDate}\\
\vspace{3in}
}

\author{\textbf{\hmwkAuthorName}}
\date{} % Insert date here if you want it to appear below your name

%----------------------------------------------------------------------------------------

\begin{document}

\maketitle

%----------------------------------------------------------------------------------------
%	TABLE OF CONTENTS
%----------------------------------------------------------------------------------------

%\setcounter{tocdepth}{1} % Uncomment this line if you don't want subsections listed in the ToC

%%\newpage
%%\tableofcontents
\newpage

%----------------------------------------------------------------------------------------
%	PROBLEM 1
%----------------------------------------------------------------------------------------

% To have just one problem per page, simply put a \clearpage after each problem

\begin{homeworkProblem}
	\begin{enumerate}[(a)]
	\item %% sub-problem (a)
		\begin{enumerate}[(1)]
		\item \textit{(Zero-coupon term structure)}
			Since the bonds are traded at par and these are riskless, zero-coupon term-structure can be obtained by just equating present value of their cash flow to their price($ = 100$). Therefore, the following equation holds.
			\begin{equation*}
				100 = \sum_{t = 1}^{T} \frac{\text{\textit{Cash flows at time t}}}{(1 + r_{0, t})^t}
			\end{equation*}
			In order to obtain the whole term-structure, one should get $r_{0, t}$ first by equating the value of bond with maturity $t$ to present value of its cash flow first, then $r_{0, t+1}$ can be obtained by using $r_{0, t}$. It is because it is necessary to use $r_{0, t}$ to discount cash flows of bond maturing at $t+1$. Therefore, calculate $r_{0, 1}$ first.
			\begin{equation*}
				100 = \frac{100 + 100 \times 0.0133}{1 + r_{0, 1}} = \frac{100(1 + 0.0133)}{1 + r_{0, 1}}
			\end{equation*}
			It is trivial that $r_{0, 1}$ = 0.0133. Then, let's use this result to calculate $r_{0, 2}$.
			\begin{equation*}
				100 = \frac{100 \times 0.0173}{1 + 0.0133} + \frac{100 + 100 \times 0.0173}{(1 + r_{0.2})^2}
			\end{equation*}
			Then it is calculated that $r_{0, 2} = 0.01733$. Using this procedure ahead, zero-coupon term-structure from 1 year to 10 year can be obtained. Table \ref{tab:zcb termstructure} shows that the term structure of zero coupon bonds obtained by using this procedure.
			
			%% Zero-Coupon Bond Term Structures Table
			\begin{table}[h]
			\centering
			\begin{tabular}{@{}cc@{}}
			\toprule
			Maturity(years) & $r_{0, t}$(\%) \\ \midrule
			1               & 1.330          \\
			2               & 1.733          \\
			3               & 2.162          \\
			4               & 2.577          \\
			5               & 2.752          \\
			6               & 3.196          \\
			7               & 3.435          \\
			8               & 3.459          \\
			9               & 3.474          \\
			10              & 3.562          \\ \bottomrule
			\end{tabular}
			\caption{Term structure of zero coupon bonds} \label{tab:zcb termstructure}
			\end{table}
		\item \textit{(Term-structure of one-year forward rates)}
			Let $r_{0, t, t+1}$ denote a forward rate from time $t$ to $t+1$ determined at current time. Then $r_{0, t, t+1}$ can be obtained by comparing the following two strategies.
			
			\begin{enumerate}
				\item Invest \$1 to zero coupon bond maturing at time $t$, and when the bond matures, receive money and reinvest to zero coupon bond maturing at time $t+1$.
				\item Invest \$1 to zero coupon bond maturing at time $t+1$.
			\end{enumerate}
			
			Since the amount of investment at current time is equal for both strategies, by definition of forward rate, the following equation should hold.
			\begin{equation*}
			\begin{aligned}
				& (1 + r_{0, t+1})^{t+1} = (1 + r_{0, t})^t (1 + r_{0, t, t+1})	\\
				& \Rightarrow r_{0, t, t+1} = \frac{(1 + r_{0, t+1})^{t+1}}{(1 + r_{0, t})^t} - 1
			\end{aligned}
			\end{equation*}
			Therefore, since the term structure of spot rate is given above, by using this formula term structure of 1-year forward rate can be obtained. Table \ref{tab: 1-fwd termstructure} shows term structure of 1-year forward rate using the data given in assignment.
			
			%% 1-Year Forward Rate Term Structure
			\begin{table}[h]
			\centering
			\begin{tabular}{@{}cc@{}}
			\toprule
			\textit{t} & $r_{0, t, t+1}$(\%) \\ \midrule
			1 & 2.139               \\
			2 & 3.025               \\
			3 & 3.830               \\
			4 & 3.459               \\
			5 & 5.441               \\
			6 & 4.881               \\
			7 & 3.627               \\
			8 & 3.592               \\
			9 & 4.363               \\ \bottomrule
			\end{tabular}
			\caption{Term structure of 1-year forward rate} \label{tab: 1-fwd termstructure}	
			\end{table}

		\end{enumerate}
	\item %% sub-problem (b)
		A fair swap rate $s$ is a rate which makes the present value of fixed leg and that of floating leg equal. Therefore, by calculating present values of each leg separately and equating each other, fair swap rate can be obtained. Let $X$ and $T$ denote the notional principal of swap and the maturity of swap, respectively. Then using the following procedure, fair swap rate $s$ can be obtained.
		\begin{enumerate}[(1)]	%% Calculate PV of fixed and floating leg
			\item \textit{(Fixed leg)}
			Since there is no exchange of notional principal in IRS, present value of fixed leg can be obtained by solving following equation.
				\begin{equation*}
					\sum_{t = 1}^{T} \frac{sX}{(1 + r_{0, t})^t}
				\end{equation*}
			\item	\textit{(Floating leg)}
			Cash flows of floating leg can be replicated by using the following investment strategies.
			\begin{enumerate}
				\item Invest $X$ amount of dollars to zero coupon bond maturing at time 1. When the bond matures, receive money and reinvest $X$ amount of dollars to zero coupon bond maturing at time 2, and repeat the procedure until time $T-1$.
				\item Borrow zero coupon bond maturing at time $T$ with face value $X$.
			\end{enumerate}
			Using the strategies above, the initial value of floating leg is $X - \frac{X}{(1 + r_{0, T})^T}$.
		\end{enumerate}
		As mentioned above, a fair swap rate $s$ makes the present value of both position equal. Therefore, the following equation holds.
		\begin{equation*}
		\begin{aligned}
			\sum_{t = 1}^{T} \frac{sX}{(1 + r_{0, t})^t} &= X - \frac{X}{(1 + r_{0, T})^T}	\\
			\Rightarrow s &= \frac{1 - \frac{1}{(1 + r_{0, T})^T}}{\sum_{t = 1}^{T} \frac{1}{(1 + r_{0, t})^t}}		\\
			&=\frac{1 - B_{0, T}}{\sum_{t = 1}^T B_{0, t}}
		\end{aligned}
		\end{equation*}
		Since $1 - B_{0, T} = \sum_{t = 1}^T (B_{0, t-1} - B_{0, t}) = \sum_{t = 1}^T B_{0, t}(\frac{B_{0, t - 1}}{B_{0, t}} - 1) = \sum_{t = 1}^T B_{0, t} R^{f}_{0, t - 1, t}$, the equation $s = \frac{\sum_{t = 1}^T B_{0, t}R^{f}_{0, t - 1, t}}{\sum_{t = 1}^T B_{0, t}}$ also holds.
		Using the equation, fair swap rates of any maturity can be calculated if sufficient data is given. Table \ref{tab:swap rate termstructure} shows term-structure of fair swap rate using given data.
		
		%% Term structure of swap rate
		\begin{table}[h]
		\centering
		\begin{tabular}{@{}cc@{}}
		\toprule
		Maturity(year) & Swap Rate(\%) \\ \midrule
		1              & 1.330         \\
		2              & 1.730         \\
		3              & 2.150         \\
		4              & 2.550         \\
		5              & 2.720         \\
		6              & 3.130         \\
		7              & 3.350         \\
		8              & 3.380         \\
		9              & 3.400         \\
		10             & 3.480         \\ \bottomrule
		\end{tabular}
		\caption{Term structure of fair swap rate} \label{tab:swap rate termstructure}
		\end{table}
		
	\item %% sub-problem (c)
	Since I am a floating payor, the value of position is worth value of fixed leg minus value of floating leg. Let $s^*$, $X$ denote a swap rate and notional pricipal of the contract, respectively. Then the value of my position is $Xs^* \sum_{t = 1}^9 B_{0, t} - X(1 - B_{0, 9})$. Therefore, my position is worth $100 \times 0.0430 \sum_{t = 1}^9 B_{0, t} - 100(1 - B_{0, 9}) = 7.004$ million dollars. If the swap rate increases by 1 basis point(i.e. if the swap rate were 4.31\%), then by using the same formula, the value of my position would be worth 7.082 million dollars, which is 0.078 million dollars greater than the original value. In contrast, if the swap rate decreases by 1 basis point(i.e. if the swap rate were 4.29\%), then my position value would be worth 6.926 million dollars, which is 0.078 million dollars less than the original value.
	\end{enumerate}
\end{homeworkProblem}

%----------------------------------------------------------------------------------------
%	PROBLEM 2
%----------------------------------------------------------------------------------------
\begin{homeworkProblem}
	\begin{enumerate}[(a)]
	\item
		Both companies will prefer the following contracts rather than the original contracts.
		\begin{enumerate}[(1)]
		\item \textit{(Company A)}
			\begin{enumerate}
				\item Make a loan with fixed rate 12.0\%
				\item Make a swap contract with paying LIBOR and receiving 12.0\%.
			\end{enumerate}
		\item \textit{(Company B)}
			\begin{enumerate}
				\item Make a loan with floating rate LIBOR + 0.6\%.
				\item Make a swap contract with paying 12.1\% and receiving LIBOR.
			\end{enumerate}		
		\end{enumerate}
		By using the strategies above, the net rate where company A has to pay is 12.0\% + LIBOR - 12.0\% = LIBOR, which is less than the offered floating rate. Furthermore, the net rate company B has to pay is LIBOR + 0.6\% + 12.1\% - LIBOR = 12.7\%, which is also less than the offered fixed rate. Acting as an intermediary, the bank receives (LIBOR - 12.0\%) + (12.1\% - LIBOR) = 10 basis points. Therefore, since every firm makes more profit by making the contracts above, all of them will prefer these contracts rather than the original contracts.
	\end{enumerate}
\end{homeworkProblem}
\end{document}
